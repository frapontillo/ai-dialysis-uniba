% This LaTeX document was generated using the LaTeX backend of PlDoc,
% The SWI-Prolog documentation system



\section{main.pl}

\label{sec:main}

\begin{description}
    \predicate[det]{main}{2}{+Config, +SymptomID}
Start the main process with custom parameters.

\begin{arguments}
\arg{Config} & The name of the configuration file. It can be:

\begin{itemize}
    \item default, for the default 'database.properties' file
    \item ask, in order to type in the file name
    \item a proper configuration file name
\end{itemize}

 \\
\arg{Symptom} & The symptom ID to use as positive target. It can be:

\begin{shortlist}
    \item default, for the default (2)
    \item ask, in order to type in the ID
    \item a proper ID
\end{shortlist}

 \\
\end{arguments}

    \predicate[det]{main_def}{0}{}
Start \predref{main}{2} with default parameters.

    \predicate[det]{main}{0}{}
Start \predref{main}{2} and asks for configurations.

    \predicate[det]{make_doc}{0}{}
Generate the documentation in both html and tex formats.
Both documentations will be under the 'doc' folder.

    \predicate[det]{out}{0}{}
Disconnects and halts.
\end{description}

\section{database.pl}

\label{sec:database}

\begin{description}
    \predicate[semidet]{db_param}{2}{?Name, ?Value}
Holds database parameters in the form \verb$db_param(Name, Value)$.
Relevant names are: server, port, database, username, password.

    \predicate[det]{get_config_path}{1}{-ConfigPath}
Ask the user to insert the configuration file name, to be found under the 'config' folder.

\begin{arguments}
\arg{ConfigPath} & The name of the database configuration file. The file has to be in the
'.properties' format. \\
\end{arguments}

    \predicate[semidet]{read_database_params}{6}{-Driver, -Server, -Port, -Database, -User, -Password}
Retrieves all the database connection parameters from a config file whose name is entered
by the user when requested.
If the user-entered configuration file doesn't exist, fall back on 'config/database.properties'.

\begin{arguments}
\arg{Driver} & The ODBC driver name. \\
\arg{Server} & The server address. \\
\arg{Port} & The server port. \\
\arg{Database} & The database to connect to. \\
\arg{User} & The username to use. \\
\arg{Password} & The password to use. \\
\end{arguments}

    \predicate[semidet]{read_database_params}{7}{+Path, -Driver, -Server, -Port, -Database, -User, -Password}
Retrieves all the database connection parameters from a given config file.

\begin{arguments}
\arg{Path} & The database configuration file name. \\
\arg{Driver} & The ODBC driver name. \\
\arg{Server} & The server address. \\
\arg{Port} & The server port. \\
\arg{Database} & The database to connect to. \\
\arg{User} & The username to use. \\
\arg{Password} & The password to use. \\
\end{arguments}

    \predicate[det]{read_database_param}{2}{+PropertiesFile, +Row}
Read database params from an open '.properties' file at the given index.
Every read property is asserts into the Prolog in-memory database.
It always iterates until the reading is over. Then, always succeeds.

\begin{arguments}
\arg{PropertiesFile} & An open '.properties' file. \\
\arg{Row} & The row to start reading from. \\
\end{arguments}

    \predicate[det]{connect}{1}{+ConfigPath}
Connect to the database by relying on the \arg{ConfigPath} variable.

\begin{arguments}
\arg{ConfigPath} & The name of the database configuration file to be found in the
'config' folder, or:

\begin{itemize}
    \item default, use the 'config/database.properties' file
    \item ask, prompt the user for a file name
\end{itemize}

 \\
\end{arguments}

    \predicate[semidet]{connect}{6}{-Driver, -Server, -Port, -Database, -User, -Password}
Try to connect to the database by using the provided prameters.

\begin{arguments}
\arg{Driver} & The ODBC driver name. \\
\arg{Server} & The server address. \\
\arg{Port} & The server port. \\
\arg{Database} & The database to connect to. \\
\arg{User} & The username to use. \\
\arg{Password} & The password to use. \\
\end{arguments}

    \predicate[semidet]{disconnect}{0}{}
If there is an open connection, disconnects from it.

    \predicate[semidet]{is_connected}{0}{}
Check if there is a currently open connection.

    \predicate[semidet]{symptom}{3}{?ID, ?Attribute, ?Value}
Holds information for all retrieved symptoms.

\begin{arguments}
\arg{ID} & The \arg{ID} of the symptom. \\
\arg{Attribute} & The name of the attribute (can be '\arg{ID}' or 'Description'). \\
\arg{Value} & The value of the cell. \\
\end{arguments}

    \predicate[det]{clear_symptoms}{0}{}
Retract all \predref{symptoms}{3}.

    \predicate[semidet]{get_symptoms}{0}{}
Fetch and assert all \predref{symptom}{2} from the database.
Requires an open connection named dialysis_connection.

    \predicate[det]{print_symptoms}{0}{}
Print all the symptom facts in the 'ID:Description' format.

    \predicate[semidet]{update_symptoms}{0}{}
Clear (with \predref{clear_symptoms}{0}) and refresh (with \predref{get_symptoms}{0}) all symptoms.

    \predicate[semidet]{exists_symptom}{1}{+ID}
Check if the given symptom \arg{ID} exists.

\begin{arguments}
\arg{ID} & \arg{ID} of the symptom to check for. \\
\end{arguments}

    \predicate[det]{exists_symptom}{1}{-Count}
\arg{Count} the number of symptoms.

\begin{arguments}
\arg{Count} & The number of existing symptoms. \\
\end{arguments}

    \predicate[semidet]{ask_target}{0}{}
Print all existing symptoms and ask the user to choose the one to predict outcomes for.
If the user-entered symptom is not valid, the default one is selected, by calling
\predref{fallback_default_target}{0}.

    \predicate[det]{fallback_default_target}{0}{}
Fall back to the default target symptom, after printing a warning.

    \predicate[det]{update_target}{1}{+PositiveID}
Save the positive and negative symptom IDs to be used for the learning step.

    \predicate[semidet]{positive}{3}{?ID, ?Attribute, ?Value}
Hold all the positive examples info, will then be split: \predref{train_positive}{3}, \predref{test_positive}{3}.

\begin{arguments}
\arg{ID} & The \arg{ID} of the example. \\
\arg{Attribute} & The \arg{Attribute} for the cell, can be: '\arg{ID}', 'Patient', 'Center',
'PatientSex', 'PatientRace', 'PatientAge', 'SessionID', 'SessionDate',
'KTV', 'QB', 'ProgWeightLoss', 'RealWeightLoss', 'DeltaWeight',
'ProgDuration', 'RealDuration', 'DeltaDuration', 'SAPStart', 'SAPEnd',
'SAPAverage', 'DAPStart', 'DAPEnd', 'DAPAverage', 'DeltaBloodFlow',
'DeltaUF', 'SymptomID', 'Score' \\
\arg{Value} & The \arg{Value} for the cell. \\
\end{arguments}

\begin{tags}
    \tag{See also}
\predref{negative}{3}
\end{tags}

    \predicate[semidet]{negative}{3}{?ID, ?Attribute, ?Value}
Hold all the negative examples info, will then be split: \predref{train_negative}{3}, \predref{test_negative}{3}.

\begin{arguments}
\arg{ID} & The \arg{ID} of the example. \\
\arg{Attribute} & The \arg{Attribute} for the cell, can be: '\arg{ID}', 'Patient', 'Center',
'PatientSex', 'PatientRace', 'PatientAge', 'SessionID', 'SessionDate',
'KTV', 'QB', 'ProgWeightLoss', 'RealWeightLoss', 'DeltaWeight',
'ProgDuration', 'RealDuration', 'DeltaDuration', 'SAPStart', 'SAPEnd',
'SAPAverage', 'DAPStart', 'DAPEnd', 'DAPAverage', 'DeltaBloodFlow',
'DeltaUF', 'SymptomID', 'Score' \\
\arg{Value} & The \arg{Value} for the cell. \\
\end{arguments}

\begin{tags}
    \tag{See also}
\predref{positive}{3}
\end{tags}

    \predicate[det]{clear_records}{1}{+RecordName}
Clear the records from the in-memory Prolog database.

\begin{arguments}
\arg{RecordName} & The name of the record, can be positive or negative. \\
\end{arguments}

    \predicate[det]{save_records}{2}{+Statement, +RecordName}
Fetch all records from a prepared \arg{Statement}.

\begin{arguments}
\arg{Statement} & The prepared statement to fetch records from. \\
\arg{RecordName} & The name of the record to assert rows into; can be positive or negative. \\
\end{arguments}

    \predicate[semidet]{get_records}{2}{+SymptomID, +RecordName}
Get all records for a given \arg{SymptomID} and assert them with a given \arg{RecordName}.

\begin{arguments}
\arg{SymptomID} & The ID of the target attribute of the record. \\
\arg{RecordName} & The name of the record to assert rows into; can be positive or negative. \\
\end{arguments}

    \predicate[det]{print_records}{1}{RecordName}
Print all records with the given \arg{RecordName}

\begin{arguments}
\arg{RecordName} & The name of the record to assert rows into; can be positive or negative. \\
\end{arguments}

    \predicate[semidet]{update_records}{1}{+SymptomID}
Update both positive and negative examples by fetching them from the database.
The negative ID is fixed (1), but the positive can be passed to this predicate.

\begin{arguments}
\arg{SymptomID} & The ID of the symptom to be used as positive target, or:

\begin{shortlist}
    \item default, use the 2
    \item ask, let the user decide
\end{shortlist}

 \\
\end{arguments}

    \predicate[semidet]{update_records}{0}{}
Clear (\predref{clear_records}{1}) and updates (\predref{get_records}{2}) all positive and negative examples.

    \predicate[semidet]{exists_record}{2}{+RecordName, ?ID}
Check if the given record \arg{ID} was asserted with the given \arg{RecordName}.

\begin{arguments}
\arg{RecordName} & The name the record was asserted with; can be positive or negative. \\
\arg{ID} & The \arg{ID} of the record. \\
\end{arguments}

    \predicate[det]{count_records}{2}{+RecordName, ?Count}
\arg{Count} the number of existing records with a given name.

\begin{arguments}
\arg{RecordName} & The name of the record to count; can be positive or negative. \\
\arg{Count} & The number of records. \\
\end{arguments}

    \predicate[det]{count_all_examples}{1}{-Count}
Return the number of all examples in the Prolog in-memory database.
@see \predref{example}{4}

\begin{arguments}
\arg{Count} & The number of retrieved examples \\
\end{arguments}
\end{description}

\section{categories.pl}

\label{sec:categories}

\begin{description}
    \predicate[semidet]{data_type}{2}{?Attribute, ?Type}
Define the type of the attribute data.

\begin{arguments}
\arg{Attribute} & The \arg{Attribute} name. \\
\arg{Type} & The \arg{Type} of the \arg{Attribute}, can be category or number. \\
\end{arguments}

    \predicate[semidet]{target_class}{1}{?Attribute}
Holds information of the target \arg{Attribute} to predict for.

\begin{arguments}
\arg{Attribute} & The \arg{Attribute} name holding the target information (always 'SymptomID'). \\
\end{arguments}

    \predicate[semidet]{class}{2}{?Attribute, ?RangeList:list}
Holds range information for each attribute.
Note: only attributes with at least one value will be classified

\begin{arguments}
\arg{Attribute} & The \arg{Attribute} name. \\
\arg{RangeList} & The (ordered) list of ranges the attribute was split into. \\
\end{arguments}

    \predicate[det]{update_categories}{0}{}
Make ranges for all of the available attributes.
For a category: every possible value is both the start and end of the class
For a number: use ranges that span (\Sbar{}max-min\Sbar{}/10) values

    \predicate[det]{make_class}{2}{+Attribute, +Type}
Make ranges for a generic \arg{Attribute} with a given \arg{Type}.

\begin{arguments}
\arg{Attribute} & The \arg{Attribute} name. \\
\arg{Type} & The \arg{Type} of the \arg{Attribute}. \\
\end{arguments}

    \predicate[det]{get_range_span}{2}{+Difference, -Span}
Calculate \arg{Difference}/10 as the range \arg{Span}.

\begin{arguments}
\arg{Difference} & The difference between a minimum and maximum value. \\
\arg{Span} & The resulting range cardinality. \\
\end{arguments}

    \predicate[det]{generate_range}{4}{+Attribute, +Min, +Max, +Span}
Recursively generate ranges for the given \arg{Attribute}.

\begin{arguments}
\arg{Attribute} & The \arg{Attribute} to generate ranges for. \\
\arg{Min} & The current minimum value of the range. \\
\arg{Max} & The current maximum value of the range. \\
\arg{Span} & The calculated range cardinality. \\
\end{arguments}

    \predicate[semidet]{add_to_class}{3}{+Attribute, +Bottom, +Top}
Add the input range to the ranges of the \arg{Attribute} (\predref{class}{2}).

\begin{arguments}
\arg{Attribute} & The \arg{Attribute} to add the range to. \\
\arg{Bottom} & The minimum value of the range. \\
\arg{Top} & The maximum value of the range. \\
\end{arguments}

    \predicate[nondet]{is_in_range}{2}{+Value, ?Range}
Check if the given \arg{Value} is included in the given \arg{Range}.
If \arg{Range} is a category, Bottom equals Top, so only a check on Bottom = \arg{Value} is performed.
If \arg{Range} is a number, \arg{Value} must be higher or equal than Bottom and

\begin{arguments}
\arg{Value} & A value, can be a number or a category. \\
\arg{Range} & A \arg{Range} structure, as \verb$range(Bottom, Top)$. \\
\end{arguments}
\end{description}

\section{util.pl}

\label{sec:util}

\begin{description}
    \predicate[det]{measure_time}{0}{}
Reset the global timer.

    \predicate[semidet]{measure_time}{1}{-Time}
Return the time (in milliseconds) elapsed from the last call to \predref{measure_time}{0}.

\begin{arguments}
\arg{Time} & The elapsed time, in milliseconds. \\
\end{arguments}

    \predicate[nondet]{timer_time}{2}{?Name, ?Time}
Holds information for existing timers.

\begin{arguments}
\arg{TimerName} & The name of the timer. \\
\arg{TimerStart} & The date (in seconds, UNIX epoch time) when the timer was started. \\
\end{arguments}

    \predicate[semidet]{timer_start}{1}{+Name}
Start a new timer with a given name.
If another timer with the given name exists, an error is raised.

\begin{arguments}
\arg{Name} & The name of the new timer. \\
\end{arguments}

    \predicate[semidet]{timer_get}{2}{+Name, -Elapsed}
Get the elapsed time from a given timer.

\begin{arguments}
\arg{Name} & The name of the timer. \\
\arg{Elapsed} & The number of elapsed seconds since the timer was started. \\
\end{arguments}

    \predicate[semidet]{timer_stop}{2}{+Name, -Elapsed}
Stop and destroy a given timer.

\begin{arguments}
\arg{Name} & The name of the timer to be stopped. \\
\arg{Elapsed} & The number of elapsed seconds since the timer was started. \\
\end{arguments}

    \predicate[semidet]{timer_must_not_exist}{1}{+Name}
Check if a timer exists and logs an error if it does.

\begin{arguments}
\arg{Name} & The name of the timer to check for. \\
\end{arguments}

    \predicate[semidet]{timer_must_exist}{1}{+Name}
Check if a timer doesn't exists and logs an error if it does not.

\begin{arguments}
\arg{Name} & The name of the timer to check for. \\
\end{arguments}

    \predicate[semidet]{format_ms}{4}{?Time, ?Minutes, ?Seconds, ?Milliseconds}
Convert a bunch of milliseconds into minutes, seconds an milliseconds.

\begin{arguments}
\arg{Time} & The time in milliseconds. \\
\arg{Minutes} & The number of minutes in \arg{Time}. \\
\arg{Seconds} & The number of seconds in \arg{Time}. \\
\arg{Milliseconds} & The number of remaining milliseconds in \arg{Time}. \\
\end{arguments}

    \predicate[semidet]{format_s}{2}{?Time, ?String}
Convert a bunch of seconds into a string representation.

\begin{arguments}
\arg{Time} & The time in seconds. \\
\arg{Minutes} & The string representation of \arg{Time}, in the '\{M\}m \{S\}s \{MS\}ms' format. \\
\end{arguments}

    \predicate[semidet]{format_ms}{2}{?Time, ?String}
Convert a bunch of seconds into a string representation.

\begin{arguments}
\arg{Time} & The time in milliseconds. \\
\arg{Minutes} & The string representation of \arg{Time}, in the '\{M\}m \{S\}s \{MS\}ms' format. \\
\end{arguments}

    \predicate[det]{println}{1}{+Element}
Print something without a separator.

\begin{arguments}
\arg{Element} & The element to print. \\
\end{arguments}

\begin{tags}
    \tag{See also}
\predref{println}{2}.
\end{tags}

    \predicate[semidet]{list_min}{2}{+List:list, ?Min}
Get the minimum element from a numeric list.

\begin{arguments}
\arg{List} & The list to look the minimum into. \\
\arg{Min} & The minimum element of the list. \\
\end{arguments}

    \predicate[semidet]{list_max}{2}{+List:list, ?Max}
Get the maximum element from a numeric list.

\begin{arguments}
\arg{List} & The list to look the minimum into. \\
\arg{Max} & The maximum element of the list. \\
\end{arguments}

    \predicate[semidet]{list_most_common}{3}{+List:list, ?Element, ?Count}
Get the most common \arg{Element} from a \arg{List}, returning the \arg{Count}.

\begin{arguments}
\arg{List} & The list to look into. \\
\arg{Element} & The most common element of the list. \\
\arg{Count} & The number of times the most common element was found in the list. \\
\end{arguments}

    \predicate[semidet]{index_of}{3}{+List:list, +Element, ?Index}
Get the position of an element in a list.

\begin{arguments}
\arg{List} & The list to look into. \\
\arg{Element} & The element to find the position for. \\
\arg{Index} & The index of the element in the list. \\
\end{arguments}

    \predicate[det]{list_push}{3}{+List:list, +Element, -ResultingList:list}
Add an element to the end of a list.

\begin{arguments}
\arg{List} & The list to append the element to. \\
\arg{Element} & The element to be appended. \\
\arg{ResultingList} & The resulting list. \\
\end{arguments}

    \predicate[det]{list_unshift}{3}{+List:list, +Element, -ResultingList:list}
Add an element to the top of a list.

\begin{arguments}
\arg{List} & The list to unshift the element to. \\
\arg{Element} & The element to be unshifted. \\
\arg{NewList} & The resulting list. \\
\end{arguments}

    \predicate[det]{list_append}{3}{+A, +B, -ResultingList:list}
Concat two generic elements in one list. Both elements can be lists or atoms, indipendently.

\begin{arguments}
\arg{A} & The first element (atom or list). \\
\arg{B} & The second element (atom or list). \\
\arg{List} & The resulting list. \\
\end{arguments}

    \predicate[det]{log2}{2}{+Expr, -R}
Calculate the base-2 logarithm of \arg{Expr}.

\begin{arguments}
\arg{Expr} & The expression to calculate the log2 for. \\
\arg{R} & The result of the calculation. \\
\end{arguments}
\end{description}

\section{learner.pl}

\label{sec:learner}

\begin{description}
    \predicate[semidet]{example}{4}{Kind, ID, Attribute, Value}
Accessory predicate to access any kind of example (both training or testing) by specifying
the name the record was asserted with (negative or positive).

\begin{arguments}
\arg{Kind} & The name the record was asserted with; can be positive or negative. \\
\arg{ID} & The \arg{ID} of the example. \\
\arg{Attribute} & The \arg{Attribute} of the example. \\
\arg{Value} & The \arg{Value} of the example. \\
\end{arguments}

    \predicate[det]{split_examples}{2}{+FoldCount, +TestFold}
Split all examples between \predref{train_examples}{1} and \predref{test_examples}{1}.

\begin{arguments}
\arg{FoldCount} & The number of total fold to split the examples between. \\
\arg{TestFold} & The index of the testing fold at the current step. \\
\end{arguments}

    \predicate[det]{split_examples}{5}{+PositiveNegative, +FoldCount, +TestFold, -TrainExamples, -TestExamples}
Split positive or negative examples in two different lists: training and testing, according
to the fold number and the current test fold.
If \arg{FoldCount} and \arg{TestFold} equal to 1, simulate the testing and training set as being the same.

\begin{arguments}
\arg{PositiveNegative} & The classification of the example, can be positive or negative. \\
\arg{FoldCount} & The number of total fold to split the examples between. \\
\arg{TestFold} & The index of the testing fold at the current step. \\
\arg{TrainExamples} & The generated training examples list of IDs. \\
\arg{TestExamples} & The generated testing examples list of IDs. \\
\end{arguments}

    \predicate[semidet]{train_examples}{1}{-List:list}
Return the list of current training examples.

\begin{arguments}
\arg{List} & The list of training examples. \\
\end{arguments}

    \predicate[nondet]{train_example}{4}{Kind, ID, Attribute, Value}
Access any kind of training example by specifying the name the record was asserted with
(negative or positive).

\begin{arguments}
\arg{Kind} & The name the record was asserted with; can be positive or negative. \\
\arg{ID} & The \arg{ID} of the example. \\
\arg{Attribute} & The \arg{Attribute} of the example. \\
\arg{Value} & The \arg{Value} of the example. \\
\end{arguments}

    \predicate[semidet]{test_examples}{1}{-List:list}
Return the list of current testing examples.

\begin{arguments}
\arg{List} & The list of testing examples. \\
\end{arguments}

    \predicate[nondet]{test_example}{4}{Kind, ID, Attribute, Value}
Access any kind of testing example by specifying the name the record was asserted with
(negative or positive).

\begin{arguments}
\arg{Kind} & The name the record was asserted with; can be positive or negative. \\
\arg{ID} & The \arg{ID} of the example. \\
\arg{Attribute} & The \arg{Attribute} of the example. \\
\arg{Value} & The \arg{Value} of the example. \\
\end{arguments}

    \predicate[nondet]{train_positive}{3}{?ID, ?Attribute, ?Value}
Hold positive training examples information.

\begin{arguments}
\arg{ID} & The \arg{ID} of the example. \\
\arg{Attribute} & The \arg{Attribute} for the cell, can be: '\arg{ID}', 'Patient', 'Center',
'PatientSex', 'PatientRace', 'PatientAge', 'SessionID', 'SessionDate',
'KTV', 'QB', 'ProgWeightLoss', 'RealWeightLoss', 'DeltaWeight',
'ProgDuration', 'RealDuration', 'DeltaDuration', 'SAPStart', 'SAPEnd',
'SAPAverage', 'DAPStart', 'DAPEnd', 'DAPAverage', 'DeltaBloodFlow',
'DeltaUF', 'SymptomID', 'Score' \\
\arg{Value} & The \arg{Value} for the cell. \\
\end{arguments}

\begin{tags}
    \tag{See also}
\predref{negative}{3}
\end{tags}

    \predicate[nondet]{train_negative}{3}{?ID, ?Attribute, ?Value}
Hold negative training examples information.

\begin{arguments}
\arg{ID} & The \arg{ID} of the example. \\
\arg{Attribute} & The \arg{Attribute} for the cell, can be: '\arg{ID}', 'Patient', 'Center',
'PatientSex', 'PatientRace', 'PatientAge', 'SessionID', 'SessionDate',
'KTV', 'QB', 'ProgWeightLoss', 'RealWeightLoss', 'DeltaWeight',
'ProgDuration', 'RealDuration', 'DeltaDuration', 'SAPStart', 'SAPEnd',
'SAPAverage', 'DAPStart', 'DAPEnd', 'DAPAverage', 'DeltaBloodFlow',
'DeltaUF', 'SymptomID', 'Score' \\
\arg{Value} & The \arg{Value} for the cell. \\
\end{arguments}

\begin{tags}
    \tag{See also}
\predref{positive}{3}
\end{tags}

    \predicate[nondet]{test_positive}{3}{?ID, ?Attribute, ?Value}
Hold positive testing examples information.

\begin{arguments}
\arg{ID} & The \arg{ID} of the example. \\
\arg{Attribute} & The \arg{Attribute} for the cell, can be: '\arg{ID}', 'Patient', 'Center',
'PatientSex', 'PatientRace', 'PatientAge', 'SessionID', 'SessionDate',
'KTV', 'QB', 'ProgWeightLoss', 'RealWeightLoss', 'DeltaWeight',
'ProgDuration', 'RealDuration', 'DeltaDuration', 'SAPStart', 'SAPEnd',
'SAPAverage', 'DAPStart', 'DAPEnd', 'DAPAverage', 'DeltaBloodFlow',
'DeltaUF', 'SymptomID', 'Score' \\
\arg{Value} & The \arg{Value} for the cell. \\
\end{arguments}

\begin{tags}
    \tag{See also}
\predref{negative}{3}
\end{tags}

    \predicate[nondet]{test_negative}{3}{?ID, ?Attribute, ?Value}
Hold negative testing examples information.

\begin{arguments}
\arg{ID} & The \arg{ID} of the example. \\
\arg{Attribute} & The \arg{Attribute} for the cell, can be: '\arg{ID}', 'Patient', 'Center',
'PatientSex', 'PatientRace', 'PatientAge', 'SessionID', 'SessionDate',
'KTV', 'QB', 'ProgWeightLoss', 'RealWeightLoss', 'DeltaWeight',
'ProgDuration', 'RealDuration', 'DeltaDuration', 'SAPStart', 'SAPEnd',
'SAPAverage', 'DAPStart', 'DAPEnd', 'DAPAverage', 'DeltaBloodFlow',
'DeltaUF', 'SymptomID', 'Score' \\
\arg{Value} & The \arg{Value} for the cell. \\
\end{arguments}

\begin{tags}
    \tag{See also}
\predref{positive}{3}
\end{tags}

    \predicate[semidet]{complete_set}{1}{?CompleteSet:list}
Get the complete set of training example IDs as a list.

\begin{arguments}
\arg{CompleteSet} & The list of all the training example IDs. \\
\end{arguments}

    \predicate[semidet]{entropy}{2}{+IncludedValues, -Entropy}
Calculate the entropy of a given list of training examples (by IDs).

\begin{arguments}
\arg{IncludedValues} & A list of the example IDs to be included when considering the entropy
calculus. The resulting examples will be intersected with the positive
and negative examples.
If you don't want to filter anything, pass in all of the example IDs:

\begin{code}
findall(ID, (train_example(_, ID, 'ID', ID),
train_example(_, ID, 'DeltaWeight', 1)), IncludedValues).
\end{code}

 \\
\arg{Entropy} & The resulting entropy of the examples passed by IDs. \\
\end{arguments}

    \predicate[semidet]{entropy}{1}{-Entropy}
Calculate the entropy of the whole set of training examples. This is a shortcut clause for:

\begin{code}
findall(ID, train_example(ID,'ID',ID), List), entropy(List, Entropy).
\end{code}

\begin{arguments}
\arg{Entropy} & The resulting entropy of the whole set of training examples. \\
\end{arguments}

    \predicate[semidet]{best_attribute}{3}{+Set:list, +Attributes:list, -Attribute}
Select the best attribute from the given lists of attributes and examples,
using the information gain measure.

\begin{arguments}
\arg{Set} & The list of example IDs to calculate the best attribute for. \\
\arg{Set} & The list of attributes to select the best attribute from. \\
\arg{Attribute} & The best attribute for the given set. \\
\end{arguments}

    \predicate[semidet]{best_attribute}{1}{-Attribute}
Select the best attribute from the whole set of attributes and training examples.
This is a shortcut clause for:

\begin{code}
findall(ID, train_example(ID,'ID',ID), Examples),
findall(Attribute, class(Attribute, _), Attributes),
best_attribute(Examples, Attributes, Entropy).
\end{code}

\begin{arguments}
\arg{Attribute} & The best attribute for all of the exmmples and attributes. \\
\end{arguments}

    \predicate[semidet]{info_gain}{3}{+Set:list, +Attribute, -InfoGain}
Calculate the Information Gain for a set of training exampes and a given attribute.

\begin{arguments}
\arg{Set} & A list of IDs to be included when considering the info gain calculus. \\
\arg{Attribute} & The attribute to calculate the information gain for. \\
\arg{InfoGain} & The calculated information gain. \\
\end{arguments}

    \predicate[det]{info_gain}{2}{+Attribute, -InfoGain}
Calculate the Information Gain for all training examples and one attribute.
This is a shortcut clause for:

\begin{code}
findall(ID, train_example(ID,'ID',ID), Examples),
info_gain(CompleteSet, Attribute, InfoGain).
\end{code}

\begin{arguments}
\arg{Attribute} & The attribute to calculate the information gain for. \\
\arg{InfoGain} & The calculated information gain. \\
\end{arguments}

    \predicate[semidet]{partial_info_gain}{4}{+Set:list, +Attribute, +Range, -PartialInfoGain}
Calculate a partial value used to compute the info gain for a given attribute.

\begin{arguments}
\arg{Set} & The list of example IDs to calculate the value for. \\
\arg{Attribute} & The attribute name to calculate the value for. \\
\arg{Range} & The specific \verb$range(Bottom, Top)$ for the given \arg{Attribute}. \\
\arg{PartialInfoGain} & The partial value to be used to compute the whole \arg{Attribute} Info Gain. \\
\end{arguments}

    \predicate[det]{clean_set}{3}{+Set:list, +Attribute, -CleanSet}
Clean the given list of example IDs (doesn't matter if training or testing) from '\$null\$' values.

\begin{arguments}
\arg{Set} & The set of example IDs to clean. \\
\arg{Attribute} & The attribute whose \$null\$ value must be deleted. \\
\arg{CleanSet} & The \arg{CleanSet}, a list whose \arg{Attribute} does not have '\$null\$'' values. \\
\end{arguments}

    \predicate[det]{partition_examples}{4}{+InExamples:list, +Attribute, +Range, -OutExamples}
Partition a list of example IDs by analyzing an attribute in a given range.

\begin{arguments}
\arg{InExamples} & List of example IDs to analyze and filter. \\
\arg{Attribute} & The attribute to filter on. \\
\arg{Range} & The range to filter with. \\
\arg{OutExamples} & List of example IDs to return. \\
\end{arguments}

    \predicate[semidet]{node}{4}{?NodeName, ?ParentNode, ?SplitAttribute, ?SplitRange}
Holds the node name information, the parent node, and the splitting attribute and range.

    \predicate[semidet]{node_label}{2}{?Node, ?Label}
Holds the node name information, the parent node, and the splitting attribute and range.

    \predicate[det]{learn_please}{0}{}
Start the learning process:

\begin{enumerate}
    \item partition the \predref{positive}{3} data set in 10 folds
    \item partition the \predref{negative}{3} data set in 10 folds
    \item for every generated fold:
a. start the learning phase
b. start the testing phase
\end{enumerate}

    \predicate[nondet]{test_step}{2}{?Step, ?List}
Holds information about a particular step run.

\begin{arguments}
\arg{Step} & The step the error rate was calculated at. \\
\arg{List} & \arg{List} containing the following:

\begin{shortlist}
    \item \verb$n(AllNegatives)$
    \item \verb$p(AllPositives)$
    \item \verb$rules(GeneratedRules)$
    \item \verb$tn(TrueNegatives)$
    \item \verb$fn(FalseNegatives)$
    \item \verb$tp(TruePositives)$
    \item \verb$fp(FalsePositives)$
    \item \verb$true_pos_rate(TruePosRate)$
    \item \verb$true_neg_rate(TrueNegRate)$
    \item \verb$false_pos_rate(FalsePosRate)$
    \item \verb$false_neg_rate(FalseNegRate)$
    \item \verb$precision(Precision)$
    \item \verb$recall(Recall)$
\end{shortlist}

 \\
\end{arguments}

\begin{tags}
    \tag{See also}
\predref{test}{1}.
\end{tags}

    \predicate[nondet]{test_final}{1}{?List}
Holds information about the final testing process.

\begin{arguments}
\arg{List} & \arg{List} containing the following:

\begin{shortlist}
    \item \verb$n(AllNegatives)$
    \item \verb$p(AllPositives)$
    \item \verb$rules(GeneratedRules)$
    \item \verb$tn(TrueNegatives)$
    \item \verb$fn(FalseNegatives)$
    \item \verb$tp(TruePositives)$
    \item \verb$fp(FalsePositives)$
    \item \verb$true_pos_rate(TruePosRate)$
    \item \verb$true_neg_rate(TrueNegRate)$
    \item \verb$false_pos_rate(FalsePosRate)$
    \item \verb$false_neg_rate(FalseNegRate)$
    \item \verb$precision(Precision)$
    \item \verb$recall(Recall)$
\end{shortlist}

 \\
\end{arguments}

\begin{tags}
\mtag{See also}- \predref{test_step}{2}. \\- \predref{test}{0}.
\end{tags}

    \predicate[nondet]{is_positive}{2}{+ID, ?LearningStep}
Check if a given \predref{example}{4} \arg{ID} is positive according to an optionally provided \arg{LearningStep}.

\begin{arguments}
\arg{ID} & The \arg{ID} of the \predref{example}{3} to check for. \\
\arg{LearningStep} & The step number of the learning process. \\
\end{arguments}

\begin{tags}
    \tag{See also}
\predref{check_condition_list}{3}
\end{tags}

    \predicate[semidet]{check_positive}{2}{+ID, ?LearningStep}
Semi-deterministic version of \predref{is_positive}{2}. If there is at least one \predref{is_positive}{2} that
satisfies the current \arg{ID}, succeed; otherwise, fail.

\begin{arguments}
\arg{ID} & The \arg{ID} of the \predref{example}{3} to check for. \\
\arg{LearningStep} & The step number of the learning process. \\
\end{arguments}

\begin{tags}
    \tag{See also}
\predref{is_positive}{2}
\end{tags}

    \predicate[semidet]{check_final_positive}{1}{+ID}
Final one-time check for the complete and purged set of rules.
If there is at least one \predref{final_positive}{2} that satisfies the current \arg{ID}, succeed;
otherwise, fail.

\begin{arguments}
\arg{ID} & The \arg{ID} of the \predref{example}{3} to check for. \\
\end{arguments}

\begin{tags}
    \tag{See also}
\predref{is_positive}{2}
\end{tags}

    \predicate[nondet]{final_positive}{2}{?ID, ?StepList}
Check if a given \predref{example}{4} \arg{ID} is positive. If the example is positive, \arg{StepList} will be
instantiated with a list of steps that generated the rule that classifies the example
as positive.

\begin{arguments}
\arg{ID} & The \arg{ID} of the \predref{example}{3} to check for. \\
\arg{StepList} & A list of step indexes where the satisfactory rule was asserted in. \\
\end{arguments}

\begin{tags}
    \tag{See also}
\predref{is_positive}{2}, \predref{check_condition_list}{3}
\end{tags}

    \predicate[det]{purge_rules}{0}{}
From the list of rules generated at each step of the learning process, assert in the Prolog
memory a set of the same rules, but without any duplicate.
A duplicate rule is considered a \predref{is_positive}{2} rule with the same body.
Rules will be asserted as \predref{final_positive}{2}, where the first argument is the ID of the example
to check for and the second argument is the list of steps where the rule was found (duplicated).

    \predicate[det]{test}{1}{+Step}
Test all \predref{test_positive}{3} and \predref{test_negative}{3} and calculates several useful information.
The calculated information is asserted as \predref{test_step}{2}.
In the end, it prints a report of the run.

\begin{arguments}
\arg{Step} & The learning step, defines the particular rules to test against. \\
\end{arguments}

\begin{tags}
    \tag{See also}
\predref{test_step}{2}
\end{tags}

    \predicate[det]{test}{0}{}
Test all \predref{positive}{3} and \predref{negative}{3} and calculates several useful information.
The calculated information is asserted as \predref{test_final}{1}.

\begin{tags}
    \tag{See also}
\predref{test_final}{1}
\end{tags}

    \predicate[det]{learn}{1}{+Step}
Start the learning process using given sets of training and testing examples by adding
\predref{node}{4} and \predref{node_label}{2} clauses to the database.

\begin{arguments}
\arg{TotalFolds} & The number of folds to split the examples into. \\
\arg{Step} & The current learning step. \\
\end{arguments}

    \predicate[det]{c45}{3}{+Node, +Examples:list, +Attributes:list}
Apply the C4.5 algorithm to a given node, that will be split according to the training examples
passed and the available attributes still left.

\begin{arguments}
\arg{Node} & The node to build. \\
\arg{Examples} & The training examples. \\
\arg{Attributes} & The list of attributes to be tested. \\
\end{arguments}

    \predicate[semidet]{print_le_tree}{0}{}
Print the learnt tree, if there is a \verb$node('root', root, Root, root)$.

    \predicate[semidet]{print_le_branch}{2}{+Node, +NestLevel}
Print a given node with a nest level that decides how much left space
the node representation must have.

A node will be printed with:

\begin{itemize}
    \item a check character, if the node is terminal and classifies positive examples
    \item a uncheck character, if the node is terminal and doesn't classify positive examples
    \item a down arrow, if the node is not terminal
\end{itemize}

\begin{arguments}
\arg{Node} & The node to print. \\
\arg{NestLevel} & The nesting level to be used. \\
\end{arguments}

    \predicate[semidet]{gen_all_the_rulez}{1}{+Step}
For each \predref{node_label}{2} with a positive outcome, generate the corresponding rule
by going up from the leaf to the tree root node.
The rule will be \predref{is_positive}{2}.

\begin{arguments}
\arg{Step} & The current learning step. \\
\end{arguments}

    \predicate[semidet]{gen_rule}{2}{+Node, +Step}
Generate the rule for the corresponding input \arg{Node}.

\begin{arguments}
\arg{Node} & The leaf \arg{Node} that holds the rule information. \\
\arg{Step} & The current learning step. \\
\end{arguments}

    \predicate[semidet]{condition}{2}{?Attribute, ?Range}
Holds information about a condition of success for an \arg{Attribute} in a given \arg{Range}.

\begin{arguments}
\arg{Attribute} & The attribute to test the condition onto. \\
\arg{Range} & The range to be used for the test. \\
\end{arguments}

    \predicate[semidet]{get_rule_list}{3}{+Node, +PrevList:list, -List:list}
Builds a list of \verb$condition(Attribute, Range)$ given a node, concatenating
the conditions to the given \arg{PrevList} (can be empty).

\begin{arguments}
\arg{Node} & The \arg{Node} to build the condition list for. \\
\arg{PrevList} & The temporary list for recursion. \\
\arg{List} & The list of conditions to return. \\
\end{arguments}

    \predicate[semidet]{ensure_not_null_conditions}{2}{+ID, +List:list}
Succeed if:

\begin{itemize}
    \item the \arg{List} is empty
    \item there is at least one condition Attribute that \predref{example}{4} with the given \arg{ID} has not null
\end{itemize}

Otherwise, it fails.

\begin{arguments}
\arg{ID} & The \arg{ID} of the \predref{example}{4} to check. \\
\arg{List} & The list of \predref{condition}{2} to loop through. \\
\end{arguments}

    \predicate[semidet]{check_condition_list}{2}{+ID, +List:list}
Check if a given example with an \arg{ID} matches all the conditions in the input list.

\begin{arguments}
\arg{ID} & The \arg{ID} of the example. \\
\arg{List} & The \arg{List} of \predref{condition}{2}. \\
\end{arguments}

    \predicate[det]{get_conditions_from_list}{2}{+Conditions:list, -Condition}
Generate a set of Prolog conjunctives from a list of \predref{condition}{2}.

\begin{arguments}
\arg{List} & The list of \predref{condition}{2}. \\
\arg{Set} & The set of Prolog conjunctives. \\
\end{arguments}

    \predicate[det]{print_report}{1}{+Elapsed}
Print a detailed report of the learning algorithm and of the executed tests.

\begin{arguments}
\arg{Elapsed} & The number of seconds the algorithm has taken to complete \\
\end{arguments}

    \predicate[det]{save_log}{1}{+Elapsed}
Save a log .csv file: 'runs/log_\{ID\}.csv'.
The 'runs' folder must exist.

The file will list, for each run:

\begin{shortlist}
    \item the symptom ID (always the same for each file)
    \item the run index
    \item the number of generated rules
    \item the true positive rate
    \item the false positive rate
    \item the precision
    \item the recall
    \item the F-Measure
\end{shortlist}

The last line contains:

\begin{itemize}
    \item the time of total execution (in seconds)
    \item the number of positive examples (including both training and testing examples)
    \item the final number of generated rules
    \item the final true positive rate
    \item the final false positive rate
    \item the final precision
    \item the final recall
    \item the final F-Measure
\end{itemize}

\begin{arguments}
\arg{Elapsed} & Total time of execution, in seconds. \\
\end{arguments}

    \predicate[det]{save_rules}{0}{}
Save all generated rules in the prolog file: 'runs/rules_\{ID\}.pl'.
The 'runs' folder must exist.
\end{description}

\section{log.pl}

\label{sec:log}

\begin{description}
    \predicate[nondet]{log_level}{4}{?Priority, ?Name, ?StringRepr, ?OptList:list}
Holds information for every logging level.

\begin{arguments}
\arg{Priority} & The priority value, the higher the better. \\
\arg{Name} & The logging level name. \\
\arg{StringRepr} & The string representation of the log level, to be used for printing. \\
\arg{OptList} & The options for color printing the level. \\
\end{arguments}

    \predicate[semidet]{log_level}{1}{+Level}
Set the log level, any level lesser than this will not be printed.
This is a setter that checks that the input parameter is instantiated and that the specified
level exists.

\begin{arguments}
\arg{Level} & The level to set the logging information to. \\
\end{arguments}

    \predicate[semidet]{log_level}{1}{-Level}
Return the current log level, checking that the input parameter is an unbound variable that can
be instantiated.

\begin{arguments}
\arg{Level} & The current minumum log level. \\
\end{arguments}

    \predicate[nondet]{can_log}{1}{?Level}
Checks if a specified level can be currently printed.
Succeed if the input log has a higher or equal priority than the current minimum level.
Fails otherwise.

\begin{arguments}
\arg{Level} & The level to be checked. \\
\end{arguments}

    \predicate[det]{log_at}{3}{+Level, +Tag, +Log}
\arg{Log} a message at a certain level with a given tag.

\begin{arguments}
\arg{Level} & The level to log the message with, choose between verbose, debug, info,
warn, error, assert. \\
\arg{Tag} & The tag to use as the log header. \\
\arg{Log} & The actual log message. \\
\end{arguments}

    \predicate[det]{log_at}{2}{+Level, +Log}
\arg{Log} a message at a certain level without a tag.
Shortcut method to avoid the logging of the tag.

\begin{tags}
    \tag{See also}
\predref{log_at}{3}.
\end{tags}

    \predicate[det]{log_tag}{3}{+Level, +Tag, +Max}
Log a tag with a certain level (used for the color) and with a character limit, beyond which
ellipsis are added.

\begin{arguments}
\arg{Level} & The level to log the tag with, choose between verbose, debug, info,
warn, error, assert. \\
\arg{Tag} & The tag to print. \\
\arg{Max} & The maximum tag length. \\
\end{arguments}

    \predicate[det]{log_v}{2}{+Tag, +Log}
\arg{Log} a tag and a message at verbose level.

\begin{arguments}
\arg{Tag} & The tag to be used in the logging entry. \\
\arg{Log} & The log message to be printed. \\
\end{arguments}

    \predicate[det]{log_d}{2}{+Tag, +Log}
\arg{Log} a tag and a message at debug level.

\begin{arguments}
\arg{Tag} & The tag to be used in the logging entry. \\
\arg{Log} & The log message to be printed. \\
\end{arguments}

    \predicate[det]{log_i}{2}{+Tag, +Log}
\arg{Log} a tag and a message at info level.

\begin{arguments}
\arg{Tag} & The tag to be used in the logging entry. \\
\arg{Log} & The log message to be printed. \\
\end{arguments}

    \predicate[det]{log_w}{2}{+Tag, +Log}
\arg{Log} a tag and a message at warn level.

\begin{arguments}
\arg{Tag} & The tag to be used in the logging entry. \\
\arg{Log} & The log message to be printed. \\
\end{arguments}

    \predicate[det]{log_e}{2}{+Tag, +Log}
\arg{Log} a tag and a message at error level.

\begin{arguments}
\arg{Tag} & The tag to be used in the logging entry. \\
\arg{Log} & The log message to be printed. \\
\end{arguments}

    \predicate[det]{log_wtf}{2}{+Tag, +Log}
\arg{Log} a tag and a message at assert level.

\begin{arguments}
\arg{Tag} & The tag to be used in the logging entry. \\
\arg{Log} & The log message to be printed. \\
\end{arguments}

    \predicate[det]{log_v}{1}{+Log}
\arg{Log} a message at verbose level.

\begin{arguments}
\arg{Log} & The log message to be printed. \\
\end{arguments}

    \predicate[det]{log_d}{1}{+Log}
\arg{Log} a message at debug level.

\begin{arguments}
\arg{Log} & The log message to be printed. \\
\end{arguments}

    \predicate[det]{log_i}{1}{+Log}
\arg{Log} a message at info level.

\begin{arguments}
\arg{Log} & The log message to be printed. \\
\end{arguments}

    \predicate[det]{log_w}{1}{+Log}
\arg{Log} a message at warn level.

\begin{arguments}
\arg{Log} & The log message to be printed. \\
\end{arguments}

    \predicate[det]{log_e}{1}{+Log}
\arg{Log} a message at error level.

\begin{arguments}
\arg{Log} & The log message to be printed. \\
\end{arguments}

    \predicate[det]{log_wtf}{1}{+Log}
\arg{Log} a message at assert level.

\begin{arguments}
\arg{Log} & The log message to be printed. \\
\end{arguments}

    \predicate[det]{test_log}{0}{}
Test the logging system by printing some messages.
\end{description}

